\newpage
\section{Theoretical background}\label{theory}

\subsection{Variational problem}
We recall the strong formulation of the general problem: on a polygonal domain $\Omega \subset \R^d$, find a solution $u$ of 
\begin{equation}\label{strong}
-\Delta u = f \text{ in } \Omega, \text{ and } u = g \text{ on } \Gamma = \partial \Omega,
\end{equation}
where $f$ and $g$ are given functions defined respectively in $\Omega$ and on $\Gamma$. 

Suppose that $f\in L^2(\Omega)$, $g\in H^{1/2}(\Gamma)$ and $d=1,2$ or $3$. We multiply equation (\ref{strong}) by a test function $v$, and we integrate the strong formulation of the problem. Thanks to Green formula, we obtain:
\begin{equation}\label{varform} \int_\Omega \nabla u\cdot \nabla v \, \mathrm{d}\Omega- \int_\Gamma (\nabla u \cdot \vec{n}) v \, \mathrm{d}\Gamma = \int_\Omega fv \, \mathrm{d}\Omega, 
\end{equation}
where $\vec{n}$ is the exterior normal vector to the surface boundary $\Gamma$. Suppose there exist a continuous lifting of the boundary data $R_g \in H^1(\Omega)$ such that  $R_g|_{\Gamma} = g$. We set $\tilde{u} = u - R_g$, so that $\tilde{u}|_\Gamma = u|_\Gamma - R_g|\Gamma = g-g = 0$, and $\nabla \tilde{u} = \nabla u - \nabla R_g$. Consequently, equation (\ref{varform}) becomes: 
\begin{equation*}
\int_\Omega \nabla \tilde{u}\cdot \nabla v \, \mathrm{d}\Omega = \int_\Omega fv \, \mathrm{d}\Omega -	 \int_\Omega \nabla R_g \cdot \nabla v \, \mathrm{d}\Omega.
\end{equation*}
In the following, we omit the tilde on $\tilde{u}$ for a matter of simplicity. The variational formulation of our problem is thus now the following: find $u$ in $V=H^1_0(\Omega)$ such that for all $v\in V$, 
\begin{equation}\label{weak}
a(u,v) = F(v),
\end{equation}
where $a : V \times V \rightarrow \R$ is such that $a(w,v) = \int_\Omega \nabla w\cdot \nabla v \, \mathrm{d}\Omega$ for all $w,v\in V$, and $F:V\rightarrow\R$ is such that for all $v\in V$, $F(v) = \int_\Omega fv \, \mathrm{d}\Omega - a(R_g, v)$.

We choose the semi-norm of $H^1(\Omega)$ to be the norm we will use for functions in $H^1_0(\Omega) = V$, thanks to Poincaré inequality. In $V$ equipped with this norm, it is easy to show that $a$ is a continuous and coercive bilinear form, and that $F$ is a linear and continuous functional. We prove here some of the main properties: for all $v,w\in V$, 
\begin{align*}
\left|F(v)\right| &= \left|\int_\Omega fv \, \mathrm{d}\Omega -\int_\Omega \nabla R_g \cdot \nabla v \, \mathrm{d} \Omega\right| \\
& \leq \left|\int_\Omega fv \, \mathrm{d}\Omega\right| + \left|\int_\Omega \nabla R_g \cdot \nabla v \, \mathrm{d} \Omega\right|\\
& \leq \|f\|_{L^2(\Omega)}\|v\|_{L^2(\Omega)} + \|\nabla R_g\|_{L^2(\Omega)}\|\nabla v\|_{L^2(\Omega)}\\
& \leq \left(\|f\|_{L^2(\Omega)}+\|\nabla R_g\|_{L^2(\Omega)}\right)\|v\|_{H_0^1(\Omega)};
\end{align*}
\begin{align}
a(v,v) &= \int_\Omega \nabla v \cdot \nabla v \, \mathrm{d} \Omega = \|v\|^2_{H_0^1(\Omega)} \nonumber \\
& \geq 1\cdot \|v\|^2_{H_0^1(\Omega)}; \label{acoercive}\\
a(w,v) &= \int_\Omega \nabla w \cdot \nabla v \, \mathrm{d} \Omega \nonumber \\
& \leq 1 \cdot \|w\|_{H_0^1(\Omega)}\|v\|_{H_0^1(\Omega)}.\label{acontinue}
\end{align} 
Cauchy-Schwarz inequality has been used to show both continuities. Thus by Lax-Milgram theorem, there exists a unique solution to the weak problem expressed in equation (\ref{weak}). 


\subsection{Discrete problem}

\subsubsection{Domain decomposition}
Consider a sequence of decompositions $\{T_h\}_h$ of $\Omega$, where $h$ corresponds to the maximum of the diameters $h_K$ of the elements $K$ of the decomposition $T_h$. Note that since $\Omega$ is a polygonal domain, then for every $h$, $T_h = \Omega$. Let us recall the definition of the diameter of an element: 

\begin{definition}[Diameter]
The \textit{diameter} of an element $K$ of a domain decomposition is defined as $h_K = \max_{x,y\in K}|x-y|$. 
\end{definition}

\noindent Since we only consider polygonal domains $\Omega$, then exact decompositions with simple polygons exist. 

\begin{definition}[Simple polygon]
A \textit{simple polygon} is an open simply connected set whose boundary is a non-intersecting line made of a finite number of straight line segments. 
\end{definition}

\noindent From now on, we suppose that every considered decomposition $T_h$ is made of a finite number of simple polygons. Note that a simple polygon is not necessarily convex, and this is what mainly characterizes VEM instead of FEM. We now introduce the following notation: 
\begin{notation}
For any set of functions $W$, we note $W|_K$ the set of functions in $W$ restricted to $K$, that is $W|_K = \{f|_K : f\in W\}$. 
\end{notation}

\noindent Let us define the bilinear form $a$ reduced to any sub-domain $D$ of $\Omega$ as follows: for all $w,v\in V$,
$$ a^D(w,v) := \int_D \nabla w \cdot \nabla v \, \mathrm{d}\Omega. $$
Since the restriction of any function of $H_0^1(\Omega)$ on a sub-domain $D$ of $\Omega$ is in $H_0^1(\Omega) \cap H^1(D)$ but is not necessarily in $H_0^1(D)$, we note that $a^D$ is also a continuous bilinear form on $V\times V$, but it is not necessarilz coercive. Moreover, if $D$ is some element $K$ of $T_h$, for all $h$ and for all $w,v\in V$, \begin{equation}\label{aKvsa}
a(w,v) = \sum_{K\in T_h} a^K(w,v).\end{equation}
In the same way, since $H_0^1(\Omega)|_K = V|_K \subset H^1(K)$, we define for all $v\in V$
\begin{align}
&|v|_{1, K} := |(v|_K)|_{H^1(K)} = \left( a^K\left( v,v \right)\right)^{1/2}, \label{def1Knorm}\\
&\text{so that } \|v\|_{H_0^1(\Omega)} = \left(\sum_{K\in T_h} |v|_{1,K}^2\right)^{1/2}.\label{H01normsplit}
\end{align} 

\noindent Furthermore, for every $h$, we will denote $E_h$ the set of edges of the decomposition $T_h$.

\subsubsection{Towards space discretization}
For every domain decomposition $T_h$, the aim of VEM is to come up with a finite subspace $V_h$ of $V$ such that:
\begin{itemize}
\item we can obtain a discretized version of problem (\ref{weak}),
\item the discretized problem has a unique solution,
\item the unique solution of the discretized problem has good approximation properties with respect to the exact solution. 
\end{itemize}

\noindent To do so, we first need to introduce a notation:
\begin{notation}
For every domain $D$ and every non-negative integer $k$, $\mathbb{P}_k(D)$ is the set of polynomials in $D$ of degree at most $k$, and $\mathbb{P}_{-1}(D)=\{0\}$.
\end{notation}

\noindent To figure out what we exactly need to build, we state and prove the following theorem: 

\begin{theorem} \label{conditions}
For each $h$, suppose that
\begin{enumerate}
\item on one hand, we have \begin{enumerate}
\item a suitable finite dimensional subspace $V_h$ of $V$;
\item a symmetric bilinear form $a_h : V_h\times V_h \rightarrow \R$, and for all $K$ in $T_h$ a bilinear form $a_h^K:V_h|_K \times V_h|_K \rightarrow \R$, such that for all $w_h,v_h\in V_h$, 
$$ a_h(w_h, v_h) = \sum_{K\in T_h} a_h^K(w_h|_K,v_h|_K).$$ For simplicity, we write $a_h^K(w_h,v_h)$ instead of $a_h^K(w_h|_K,v_h|_K)$;
\item an element $f_h \in V_h'$ where $V_h'$ is the dual of $V_h$;
\end{enumerate}
\item on the other hand, there exists an integer $k\geq 1$ such that for all $K \in T_h$,
\begin{enumerate}
\item $\mathbb{P}_k(K) \subset V_h|_K$, that is our virtual space restricted to any element of the domain decomposition contains the set of polynomials of order at most $k$;
\item(k-consistency) for all $p\in \mathbb{P}_k(K)$, for all $v_h\in V_h|_K$, the relation $a_h^K(p, v_h) = a^K(p, v_h)$ is verified;
\item (stability) there exist $\alpha^-, \alpha^+ \geq 0$ two constants independent of $h$ and of $K$ such that for all $v_h\in V_h|_K$,
$$ \alpha^-a^K(v_h,v_h)\leq a_h^K(v_h,v_h)\leq \alpha^+a^K(v_h,v_h).$$
\end{enumerate}
\end{enumerate}
Under these hypothesis, 
\begin{enumerate}
\item the following discrete problem: find $u_h\in V_h$ such that for all $v_h \in V_h$, \begin{equation}\label{discretepb}
a_h(u_h,v_h) = \langle f_h,v_h\rangle,
\end{equation} admits a unique solution;
\item for every approximations $u_I\in V_h$ and $u_\pi$ that is piecewise in $\mathbb{P}_k$, of the exact solution $u$ of problem (\ref{weak}), $$\|u-u_h\|_{H_0^1(\Omega)} \leq C\left(\|u-u_I\|_{H_0^1(\Omega)} + \|u-u_\pi\|_{H_0^1(\Omega)} + \mathcal{F}_h\right),$$
where $C=C(\alpha^-,\alpha^-)$, and $\mathcal{F}_h$ is the smallest constant such that for all $v\in V_h$, 
\begin{equation}\label{calFh}
F(v)-\langle f_h,v\rangle \leq \mathcal{F}_h\|v\|_{H^0_1(\Omega)}.
\end{equation}
\end{enumerate}
\end{theorem}
\begin{remark}
Note that here, for $f_h\in V_h'$ and $v_h \in V_h$, $\langle f_h,v_h\rangle$ designs the action of $f_h$ on $v_h$. Moreover, $u_\pi$ of statement $2$ is piecewise in $\mathbb{P}_k$ and $\mathbb{P}_k(K)$ is included in $V_h|_K$ for all $K\in T_h$ (from hypothesis $2.a$), so $u_\pi$ belongs to $\Pi_{K\in V_h} \mathbb{P}_k(K) \subset \Pi_{K\in V_h} V_h|_K = V_h \subset H_0^1(\Omega)$. This is why the norm of $u-u_\pi$ can be the $H_0^1(\Omega)$ norm. 
\end{remark}

\noindent Let us prove this theorem:
\begin{proof}
We will prove the two statements in order. The abbreviation CS is used to say that we use the Cauchy-Schwarz inequality.
\begin{enumerate}
\item We want to use Lax-Milgram theorem. 
\begin{itemize}
\item Since $V_h$ is a finite dimensional subspace of the Hilbert space $V$, then it is a closed subspace of an Hilbert space, so that $V_h$ is also an Hilbert space. We still consider the $H^1(\Omega)$-semi-norm as the $H_0^1(\Omega)$-norm, i.e. the $V$-norm, and thus as the $V_h$-norm as well.
\item By hypothesis $1.b$ and $1.c$, $a_h$ is a bilinear form and $f_h$ is a linear and continuous functional.
\item Continuity of $a_h$: first, for all $w,v\in V_h|_K$, 
\begin{align}
a_h^K(w,v) &\leq \left( a_h^K\left( w,w \right)\right)^{1/2}\left( a_h^K\left (v,v \right)\right)^{1/2} &\text{ (CS)} \nonumber \\
&\leq \alpha^+\left( a^K\left( w,w \right)\right)^{1/2}\left( a^K\left (v,v \right)\right)^{1/2} &\text{ (stability)} \nonumber \\
&= \alpha^+|u|_{1,K}|v|_{1,K}. &\text{ (definition)} \label{ahKalpha}
\end{align}
Consequently, for all $w_h, v_h \in V_h$, 
\begin{align*}
a_h(w_h,v_h) &= \sum_{K\in T_h} a_h^K(w_h,v_h) \hspace{3.2cm} \text{ (hypothesis }1.b)\\
&\leq \alpha^+\sum_{K\in T_h} |w_h|_{1,K}|v_h|_{1,K} \hspace{2.4cm}\text{ (inequality \ref{ahKalpha})}\\
&\leq \alpha^+\left( \sum_{K\in T_h}|w_h|_{1,K}^2 \right)^{1/2}\left( \sum_{K\in T_h}|v_h|_{1,K}^2 \right)^{1/2} \\
&\hspace{6.9cm} \text{ (discrete CS)}\\
&= \alpha^+ \|w_h\|_{H_0^1(\Omega)}\|v_h\|_{H_0^1(\Omega)}. \hspace{2.6cm}\text{ (equation \ref{H01normsplit})}
\end{align*}
\item Coercivity of $a_h$: for all $v_h\in V_h$, 
\begin{align*}
a_h(v_h,v_h) &= \sum_{K\in T_h} a_h^K(v_h,v_h) &\text{ (hypothesis }1.b)\\
&\geq \alpha^-\sum_{K\in T_h} a^K(v_h,v_h) &\text{ (stability)}\\
&= \alpha^- a(v_h,v_h) &\text{ (equation \ref{aKvsa})}\\
&= \alpha^- \|v_h\|_{H_0^1(\Omega)}^2. &\text{ (equation \ref{acoercive})}
\end{align*}
\end{itemize}
Consequently, by Lax-Milgram theorem, the discrete problem (\ref{discretepb}) admits one unique solution $u_h$.
\item Let $u_I\in V_h$ and $u_\pi$ piecewise in $\mathbb{P}_k$ such that they are approximations of the exact solution $u$ of problem (\ref{weak}). Let $\delta_h = u_h-u_I\in V_h$. Then, 
\begin{align*}
\alpha^-\|\delta_h\|^2_{H_0^1(\Omega)}
&\leq a_h(\delta_h,\delta_h) \text{ (coercivity of }a_h)\\
&= a_h(u_h,\delta_h) - a_h(u_I,\delta_h) \\
& \hspace{4.1cm}(a_h \text{ linear and definition of }\delta_h)\\
&= \langle f_h, \delta_h\rangle - \sum_{K\in T_h}a_h^K(u_I,\delta_h) \\
& \hspace{3.7cm}\text{(problem (\ref{discretepb}) and hypothesis }1.b)\\
&= \langle f_h, \delta_h\rangle - \sum_{K\in T_h}\left( a_h^K\left( u_I-u_\pi, \delta_h \right) + a_h^K\left( u_\pi,\delta_h\right) \right)\\
&\hspace{5.7cm} (a_h \text{ linear and }\pm u_\pi)\\
&= \langle f_h, \delta_h\rangle - \sum_{K\in T_h}\left( a_h^K\left( u_I-u_\pi, \delta_h \right) + a^K\left( u_\pi,\delta_h\right) \right)\\
&\hspace{6.7cm} (k\text{-consistency)}\\
&= \langle f_h, \delta_h\rangle - \sum_{K\in T_h}\left( a_h^K\left( u_I-u_\pi,\delta_h \right) + a^K\left( u_\pi-u, \delta_h \right) \right) \\
&\hspace{1.7cm}- a\left( u, \delta_h \right) \hspace{0.6cm} (\pm u, a^K \text{linear and equation \ref{aKvsa})}\\
&= \langle f_h, \delta_h\rangle - \sum_{K\in T_h}\left( a_h^K\left( u_I-u_\pi,\delta_h \right) + a^K\left( u_\pi-u, \delta_h \right) \right) \\
&\hspace{1.7cm}- F(\delta_h) \hspace{3.8cm}\text{(problem (\ref{weak}))}\\
&\leq |F(\delta_h) - \langle f_h, \delta_h \rangle| \\
&\hspace{0.5cm} - \sum_{K\in T_h}\left( a_h^K\left( u_I-u_\pi,\delta_h \right) + a^K\left( u_\pi-u, \delta_h \right) \right)\\
&\leq \mathcal{F}_h\|\delta_h\|_{H_0^1(\Omega)}-a_h(u_I-u_\pi,\delta_h)-a(u_\pi-u,\delta_h)\\
&\hspace{2.9cm} \text{(inequalities \ref{calFh}, \ref{aKvsa}, and hypothesis }1.b)\\
&\leq \|\delta_h\|_{H_0^1(\Omega)} \left (\mathcal{F}_h + \alpha^+\|u_I-u_\pi\|_{H_0^1(\Omega)} + \|u_\pi-u\|_{H_0^1(\Omega)} \right)\\
&\hspace{5cm} (\text{continuity of } a_h \text{ and of } a)\\
&\leq \|\delta_h\|_{H_0^1(\Omega)} \max\{\alpha^+,1\} \Big(\mathcal{F}_h + \|u_I-u_\pi\|_{H_0^1(\Omega)} \\
& \hspace{4.85cm} + \|u_\pi-u\|_{H_0^1(\Omega)} \Big).
\end{align*}
Now, let $\tilde{C} := \max\{\alpha^+,1\}>0$, then:
\begin{equation*}
\alpha^- \|\delta_h\|_{H_0^1(\Omega)}\leq \tilde{C}\left(\mathcal{F}_h + \|u_I-u_\pi\|_{H_0^1(\Omega)} + \|u_\pi-u\|_{H_0^1(\Omega)} \right)
\end{equation*}
Therefore, we finally obtain:
\begin{align*}
\|u-u_h\|_{H_0^1(\Omega)} &\leq \|u-u_I\|_{H_0^1(\Omega)} + \|\delta_h\|_{H_0^1(\Omega)} \text{ (triangle inequality)}\\ 
&\leq \left(\frac{\tilde{C}}{\alpha^-}+1\right)\left(\mathcal{F}_h + \|u_I-u_\pi\|_{H_0^1(\Omega)} + \|u_\pi-u\|_{H_0^1(\Omega)} \right).
\end{align*}
Consequently, we have proved statement $2$ with $C=\frac{\tilde{C}}{\alpha^-}$, and the theorem is showed. 
\end{enumerate}
\end{proof}

\begin{remark}
\begin{itemize} A few remarks on the theorem:
\item The integer $k\geq 1$ will correspond to the order of accuracy of the method.
\item The stability property (hypothesis $2.c$) is crucial to show the existence and uniqueness of the solution since without it, we cannot show that $a_h$ has the required properties to apply Lax-Milgram theorem.
\end{itemize}
\end{remark}

Consequently, for every integer $k\geq 1$, our aim is now to build the subspace $V_h$, the symmetric bilinear form $a_h$ and the element $f_h$ so that they approximate well the original problem and they verify the properties of the theorem.

\subsubsection{Choice of the degrees of freedom}

For this whole part, we consider a certain domain decomposition $T_h$ for some $h$, and $K$ any element of $T_h$. In $2$D, $K$ can consequently be any simple polygon, while in $3$D, $K$ can be any polyhedron. Moreover, for any geometrical domain $D$, let $n_e^D$ be the number of edges of $D$, $n_v^D$ be its number of vertices, and $n_f^D$ be its number of faces. Note that for a $1$D domain, the faces, the edges and the vertices represent the same geometrical elements. As a convention, we will thus set $n_e^D = 0$ in this case, that is only if $D$ is $1$-dimensional. We also note that for a $2$D domain, the faces and the edges represent the same geometrical elements as well. Furthermore, in $2$D, the number of edges of $D$ is equal to the number of vertices of $D$, even if here edges and vertices are two distinct geometrical entities. In $3$D, however, the number of vertices and of edges is not the same.

Let us also fix an integer $k\geq 1$. Finally, for any element $K\in T_h$, we will denote $E_K$ the set of edges of $K$ with respect to the domain decomposition $T_h$, and $F_K$ its set of faces. Note that as a consequence, $F_K=\{K\}$ if the dimension $d$ is $1$ or $2$, and $E_K=\{K\}$ if $d=1$. \\

We now define, for every face $f_K$ of $K$, the set $\mathbb{B}_k(\partial f_K)$ as the set of continuous functions on $\partial f_K$ that are \textit{polygonal} on each edge of $\partial f_K$. That is: 
$$ \mathbb{B}_k(\partial f_K) := \Big\{v\in C^0(\partial f_K): \forall e\in E_{\partial f_K}, v|_e \in \mathbb{P}_k(e)\Big\}.$$

\begin{proposition} \label{dimBk}
The space $\mathbb{B}_k(\partial f_K)$ is a linear space. Moreover, its dimension is equal to $n_v^{f_K}+n_e^{f_K}(k-1)$.
\end{proposition}

\begin{proof}
First, let us prove that it is a linear space. Let $v,w\in \mathbb{B}_k(\partial f_K)$ and $\lambda$ in $\R$. Then, since $v$ and $w$ are continuous on $\partial f_K$, then $\lambda v+w$ is also continuous on $\partial f_K$. Moreover, for all $e\in E_{\partial f_K}$, $v|_e, w|_e\in \mathbb{P}_k(e)$. Consequently, $\left(\lambda v + w\right)|_e = \lambda v|_e + w|_e \in \mathbb{P}_k(\partial f_K)$. \\
Now, let us compute the dimension of $\mathbb{B}_k(\partial f_K)$. We know that a polynomial of degree at most $k$ is uniquely determined by its values in $k+1$ points: 
\begin{itemize}
\item its values at the vertices, which gives us $n_v^{f_K}$ conditions;
\item its values at $k-1$ additional points on each edge of $f_K$, if $\dim(f_K)>1$, which gives us $n_e^{f_K}(k-1)$ additional conditions.
\end{itemize}
Therefore, the dimension of $\mathbb{B}_k(\partial f_K)$ is equal to $n_v^{f_K}+n_e^{f_K}(k-1)$. 
\end{proof}

\begin{remark}
Since $f_K$ is either a polygon (if $d=1$ or $2$) or a segment (if $d=1$), then in any case, $n_e^{f_K} = n_f^{f_K}$, so that $$\dim\left(\mathbb{B}_k\left(\partial f_K\right)\right) = n_v^{f_K}+n_e^{f_K}(k-1) = kn_v^{f_K} = kn_e^{f_K}.$$ 
We have chosen to state the proposition without using this fact to make clear the choice of degrees of freedom that will follow in this report.
\end{remark}

Now, recall that we want to build a space $V_h$ such that for all $K\in T_h$, we have $\mathbb{P}_k(K) \subset V_h|_K$. Moreover, we remark that every function $v$ belongs to $\mathbb{P}_k(K)$ if and only if $\Delta v\in \mathbb{P}_{k-2}(K)$. This is the reason why we decide to introduce the following local spaces $V^{K,k}$. To define it, we separate the case $d=1$ or $2$ to the case $d=3$. \newline 

\noindent \underline{Case 1:} $d=1$ or $2$. \\
\noindent Let us define:
$$ V^{K,k} := \Big\{ v\in H^1(K) : v|_{\partial K} \in \mathbb{B}_k(\partial K), \Delta v \in \mathbb{P}_{k-2}(K)\Big\}.$$
We recall that $\mathbb{P}_{-1} = \{0\}$. Note that if in $V^{K,k}$ we were asking $\Delta v|_K$ to belong to a lower-degree polynomial space, we would have $\mathbb{P}_k(K)\not\subset V^{K,k}$, which is bad since we want the opposite. Instead, if in $V^{K,k}$ we were asking $\Delta v|_K$ to belong to a higher-degree polynomial space, then we would still have in $V^{K,k}$ all the polynomials of degree at most $k$ as well. However, it adds some useless complexity to the finite space, i.e. we would need to determine more degrees of freedom without obtaining a better method at the end.

\begin{proposition} \label{dimVKkd12}
The dimension $N^{K,k}$ of the space $V^{K,k}$ is equal to $$N^{K,k} = n_v^K +(k-1)$$ if $d=1$, or if $d=2$, it is equal to $$N^{K,k} = n_v^K + n_e^K(k-1) + \frac{k(k-1)}{2}.$$
\end{proposition}
To have a first insight on this, let us first look at some particular cases. If $k=1$, $V^{K,1}$ is the space of harmonic functions (since for every function $v$ in $V^{K,1}, \Delta v|_K = 0$), that are linear on the edges of $K$. Consequently, these functions are completely determined by their values on the $n_v^K$ vertices of $K$, and $N^{K,1} = n_v^K$. 

If $k=2$, $V^{K,2}$ is the space of functions that have a constant Laplacian and that are linear or quadratic on the boundary of $K$. Moreover, for every constant $c$ and every boundary function $h\in \mathbb{B}_2(\partial K)$, by Lax-Milgram theorem, there is a unique function $v\in H^1(K)$ such that $\Delta v = c$ in $K$, and $v = h$ on $\partial K$. The differential problem has to be interpreted in a weak way. Consequently, $$N^{K,2} = \dim\big(\mathbb{B}_2(\partial K)\big)+1 = n_v^K + n_e^K+1.$$

We give now the proof of the general case:
\begin{proof}
For every $q\in \mathbb{P}_{k-2}(K)$, and for every $h\in \mathbb{B}_k(\partial K)$, by Lax-Milgram theorem, there is a unique function $v\in H^1(K)$ such that $\Delta v = q$ in $K$ and $v = h$ on $\partial K$, weakly. Consequently, 
\begin{align*}\dim\left(V^{K,k}\right) := N^{K,k} &= \dim\big(\mathbb{B}_k\left(\partial K\right)\big)+\dim\big(\mathbb{P}_{k-2}(K)\big)\\
&= n_v^K + n_e^K(k-1) + \binom{k-2+d}{k-2}.
\end{align*}
So if $d=1$, 
$N^{K,k} = n_v^K + n_e^K(k-1) + (k-1) = n_v^K + (k-1),$ since $n_e^K=0$, and if $d=2$, we obtain $N^{K,k} = n_v^K + n_e^K(k-1) + \frac{k(k-1)}{2}.$
\end{proof}

\noindent \underline{Case 2:} $d=3$. \\
\noindent We need intermediate steps before defining $V^{K,k}$, so let us first define the following: for all $f_k\in F_K$, 
$$V^{f_K,k} := \left\{v\in H^1(f_K) : v|_{\partial f_K} \in \mathbb{B}_k(f_K), \Delta v\in \mathbb{P}_{k-2}(f_K)\right\}.$$
Note that this is the exact same definition as the one of $V^{K,k}$ for $d=1,2$, except that this time, we define the set of functions that are $H^1$ on the faces of $K$, and not directly on the element $K$ itself. With the exact same proof as before, we consequently know that the dimension of $V^{f_K,k}$ is equal to $n_v^{f_K} + n_e^{f_K}(k-1) + \frac{k(k-1)}{2}$. 

We now enlarge this space on functions that are continuous on $\partial K$ entirely. Since $\partial K = \cup_{f_K\in F_K}f_K$, This gives us the following definition:
$$V^{\partial K,k} := \left\{v\in C^0(\partial K) : v|_{f_K}\in V^{f_K,k}, \forall f_K\in F_K\right\}.$$
To compute the dimension of this space, we have to be careful not to take multiple times the same edge into account, since each edge is shared by two faces. That is, the dimension is not the sum over the faces $f_K$ of the dimension of $V^{f_K,k}$. Instead, if we come back from the proof of Proposition \ref{dimBk}, we get:
\begin{align*}
\dim\left(V^{\partial K, k}\right) &= n_v^K + n_e^K(k-1) + \sum_{f_K\in F_K}\Big(\dim\big(\mathbb{P}_{k-2} \left(f_K\right)\big)\Big)\\
&= n_v^K + n_e^K(k-1) + n_f^K\frac{k(k-1)}{2}.
\end{align*}

We are now finally able to define the space $V^{K,k}$, similarly as in the $1$- and $2$-dimensional cases:
$$V^{K,k} = \left\{v\in H^1(K):v|_{\partial K}\in V^{\partial K,k}, \Delta v\in \mathbb{P}_{k-2}(K)\right\}.$$
To sum up briefly, $V^{K,k}$ is a finite dimensional subspace of $H^1(K)$ composed of functions that are polynomials of degree at most $k$ on each edge of $K$, and whose gradient is a polynomial of degree at most $k-2$ in the element $K$ itself and on each face of $K$. 

\begin{proposition}\label{dimVKkd3}
The dimension $N^{K,k}$ of the space $V^{K,k}$ is equal to $$N^{K,k} = n_v^K + n_e^K(k-1) + n_f^K\frac{k(k-1)}{2} + \frac{k(k^2-1)}{6}.$$
\end{proposition}
\begin{proof}
Indeed, 
\begin{align*}
N^{K,k} &= \dim\left(V^{\partial K,k}\right) +  \dim\left(\mathbb{P}_{k-2}(K)\right)\\
&= n_v^K + n_e^K(k-1) + n_f^K\frac{k(k-1)}{2} + \binom{k-2+d}{k-2} \\
&= n_v^K + n_e^K(k-1) + n_f^K\frac{k(k-1)}{2} + \frac{k(k^2-1)}{6}.
\end{align*}
\end{proof}


The dimension of the local space $V^{K,k}$ correspond to the number of degrees of freedom we have to choose to have an easy way to express any local function belonging to this space. VEM is based on the following choice. For all $v_h\in V^{K,k}$, $v_h$ is uniquely defined if we know the following values:
\begin{itemize}
\item the values of $v_h$ at each vertex of $K$. Let $\mathcal{V}^{K,k}$ be the set of such values;
\item if $d>1$, the values of $v_h$ at $k-1$ intermediate points on each edge of $K$. Let $\mathcal{E}^{K,k}$ be the set of such values;
\item if $d=3$, the moments up to order $k-2$ of $v_h$ on each face $f_K\in F_K$, that is $\frac{1}{|f_K|}\int_{f_K}m(\mathbf{x})v_h(\mathbf{x})\, \mathrm{d}\mathbf{x}$, for all $m\in \mathcal{M}_{k-2}(f_K)$, where $$\mathcal{M}_{k-2}(D) = \left\{\left(\frac{\mathbf{x}-\mathbf{x}_D}{h_D}\right)^\mathbf{s} : |\mathbf{s}|\leq k-2\right\},$$ with $\mathbf{x}_D$ being the barycenter of the geometrical element $D$, $h_D$ its diameter, $|\mathbf{s}| = \sum_{i=1}^d s_i$ with $s_i\in \N_0$, and $\mathbf{x}^\mathbf{s}=\prod_{i=1}^d x_i^{s_i},$ $x_i$ being the $i$-th coordinate of $\mathbf{x}$. Let $\mathcal{F}^{K,k}$ be the set of such values.
\item the moments up to order $k-2$ of $v_h$ in $K$, that is $\frac{1}{|K|}\int_Km(\mathbf{x})v_h(\mathbf{x})\, \mathrm{d}\mathbf{x}$, for all $m\in \mathcal{M}_{k-2}(K)$, where $\mathcal{M}_{k-2}(K)$ is defined in the previous point. Let $\mathcal{P}^{K,k}$ be the set of such values.
\end{itemize}

\noindent Note that for $k=2$, the only moment in $\mathcal{M}_{k-2}(D)$ is the mean value of $v_h$ in $D$, for every geometrical domain $D$.  Moreover, in $1$D, the degrees of freedom are only $\mathcal{V}^{K,k}\cup\mathcal{P}^{K,k}$, and in $2$D, the degrees of freedom are only $\mathcal{V}^{K,k}\cup\mathcal{E}^{K,k} \cup\mathcal{P}^{K,k}$.
We easily verify the following proposition:
\begin{proposition} \label{chosenlocaldofnb}
The total number of chosen degrees of freedom is equal to $N^{K,k}$. 
\end{proposition}
\begin{proof}
Indeed, we first see straightforward that $\mathcal{V}^{K,k}$ contains $n_v^K$ degrees of freedom, and $\mathcal{E}^{K,k}$ contains $n_e^K(k-1)$ degrees of freedom. Then, we have to distinguish the cases $d=1$, $d=2$ and $d=3$. \\

\noindent \underline{Case 1:} $d=1$. \\
\noindent The cardinality of $\mathcal{P}^{K,k}$ is equal to the cardinality of $\mathcal{M}_{k-2}(K)$, which is the number of multi-indices $\mathbf{s}$ such that $|\mathbf{s}|\leq k-2$ is equal to the dimension of $\mathbb{P}_{k-2}(K)$, that is $k-1$. Consequently, 
$$\#\left(\mathcal{V}^{K,k} \cup\mathcal{P}^{K,k}\right)=n_v^K
+(k-1)= \dim\left(V^{K,k}\right).
$$

\noindent \underline{Case 2:} $d=2$.\\
\noindent In this case, we do not consider $\mathcal{F}^{K,k}$, but we do consider $\mathcal{P}^{K,k}$. Moreover, the number of multi-indices $\mathbf{s}$ such that $|\mathbf{s}|\leq k-2$ is also equal to the dimension of $\mathbb{P}_{k-2}(K)$, but with $K$ that is now in $2$ dimensions. So that $\#\big(\mathcal{M}_{k-2}(K)\big) = \frac{k(k-1)}{2}$. Consequently, 
$$\#\left(\mathcal{V}^{K,k}\cup \mathcal{E}^{K,k} \cup\mathcal{P}^{K,k}\right)=n_v^K
+ n_e^K(k-1) + \frac{k(k-1)}{2}= \dim\left(V^{K,k}\right).
$$

\noindent \underline{Case 3:} $d=3$.\\
\noindent In this case, the cardinal of $\mathcal{P}^{K,k}$ is also equal to the number of multi-indices $\mathbf{s}$ such that $|\mathbf{s}|\leq k-2$, that is to the dimension of $\mathbb{P}_{k-2}(K)$. But this time, $K$ is in $3$ dimensions, so $\dim\big(\mathbb{P}_{k-2}(K)\big) = \frac{k(k^2-1)}{6}.$ Moreover, we also have to consider the set $\mathcal{F}^{K,k}$ of degrees of freedom. This set has cardinality $n_f^K\cdot\#\left(\mathcal{M}_{k-2}(f_K)\right) = n_f^K\frac{k(k-1)}{2}$, since $f_K$ is a $2$-dimensional geometrical object. Consequently, we obtain:
\begin{align*}
&\#\left(\mathcal{V}^{K,k}\cup \mathcal{E}^{K,k}\cup\mathcal{F}^{K,k} \cup\mathcal{P}^{K,k}\right)\\
= \hspace{1mm}&n_v^K
+ n_e^K(k-1) + n_f^K\frac{k(k-1)}{2} + \frac{k(k^2-1)}{6}\\
= \hspace{1mm}&\dim\left(V^{K,k}\right).
\end{align*}
\end{proof}

\begin{remark}\label{existvhfromdof}
On one hand, the degrees of freedom in $\mathcal{V}^{K,k}$ and in $\mathcal{E}^{K,k}$ uniquely determine a polynomial of degree at most $k$ on each edge of $K$. That is, $v_h$ is uniquely determined on the edges of $K$ by those two sets of degrees of freedom. On the other hand, $\mathcal{F}^{K,k}$ and $\mathcal{P}^{K,k}$ determine the $L^2$-projection of $v_h$ onto $\mathbb{P}_{k-2}(f_K)$ for all $f_K \in F_K$, and onto $\mathbb{P}_{k-2}(K)$, respectively. 
\end{remark}

Let us show now this important proposition:
\begin{proposition} \label{localunisolvent}
The degrees of freedom of $\mathcal{V}^{K,k}\cup\mathcal{E}^{K,k}\cup \mathcal{F}^{K,k}\cup\mathcal{P}^{K,k}$ are unisolvent for $V^{K,k}$. This means that for any values of the degrees of freedom, there exist a unique $v_h$ in $V^{K,k}$ defined from them. 
\end{proposition}
\begin{proof}
In remark \ref{existvhfromdof}, we have seen that from the degrees of freedom, we can define a $v_h$ in $V^{K,k}$. It thus remains to prove that the map that sends the degrees of freedom onto a function of $V^{K,k}$ is injective. 

Let us call $P_{k-2}^D$ the projection operator, in the $L^2(D)$-norm, onto the space $\mathbb{P}_{k-2}(D)$, for any space $D$. We thus want to show that any function $v_h\in V^{K,k}$ such that:
\begin{enumerate}
\item $v_h=0$ on $\cup_{e\in E_K} \bar{e}$, 
\item $P_{k-2}^{f_K}v_h = 0$ in $f_K$, for all $f_K\in F_K$,
\item $P_{k-2}^Kv_h = 0$ in $K$,
\end{enumerate}
is identically zero in $K$. In particular, condition $1.$ means that $v_h = 0$ on $\partial f_K$, for all $f_K\in F_K$.

To do so, we first show that $\Delta v_h|_{f_K}=0$ in $f_K$, for all $f_K\in K$. So that together with $1.$, we have the following problem for all $f_K\in F_K$: find $w_h\in V^{f_K,k}$ such that $\Delta w_h = 0$ in $f_K$ and $w_h = 0$ on $\partial f_K\subset E_K$. We can easily see that the function $w_h\equiv 0$ is solution of this problem. Moreover, by Strang theorem, the problem admits a unique solution. Consequently, the only solution is the identically zero solution in $V^{f_K,K}$. This would thus show that $v_h|_{f_K}$ is zero in $V^{f_K,K}$, for all $f_K\in F_K$. Therefore, $v_h|_{\partial K} \equiv 0$. 

Then, if we show that $\Delta v_h = 0$ in the whole element $K$, then together with the preceding result, we would have the following problem: find $v_h$ in $V^{K,k}$ such that $\Delta v_h = 0$ in $K$ and $v_h=0$ on $\partial K$. Again by Strang theorem, the problem admits a unique solution, and the identically zero function in $V^{K,k}$ is this solution. Consequently, $v_h$ is identically zero in $K$, and thus the degrees of freedom chosen above are unisolvent for $V^{K,k}$. \\

It thus remains to show the two points just mentioned. So let $f_K\in F_K$ and let $v_h\in V^{K,k}$ such that $v_h$ verifies $1.$, $2.$ and $3.$ In the following, we will write $v_h$ instead of $v_h|_{f_K}$ in order to lighten the proof. We will first solve the following auxiliary problem: $\forall q\in \mathbb{P}_{k-2}(f_K)$, find $w\in H_0^1(f_K)$ such that \begin{equation}\label{auxpb}
a^{f_K}(w,v)=\int_{f_K}qv \,\mathrm{d}f_K\text{, for all } v\in H_0^1(f_K),
\end{equation}
which can also be written as $-\Delta w=q$ in $f_K$, $w=0$ on $\partial f_K$ but it is meant in a weak form. Since we know by Lax-Milgram theorem that there exists a unique solution $w$ to this problem, we will write this solution as $w=-\Delta_{L^2(f_K)}^{-1}(q)$. Indeed, since we look at functions in $H_0^1(f_K)$, then the bilinear form $a^{f_K}$ is continuous and coercive, and the proof is similar as the one done on the bilinear form $a$ (see equation \ref{acoercive}). Now, let us define the map $R:\mathbb{P}_{k-2}(f_K)\rightarrow\mathbb{P}_{k-2}(f_K)$ such that:
$$R(q) := P_{k-2}^{f_K}\left(-\Delta_{L^2(f_K)}^{-1}\left( q\right) \right)=P_{k-2}^{f_K}(w).$$
Let us see that $R$ is injective. Indeed, for every $q\in \mathbb{P}_{k-2}(f_K)$, 
\begin{align*}
\int_{f_K} qR(q)\, \mathrm{d}f_K &= \int_{f_K} qP_{k-2}^{f_K}\left(-\Delta_{L^2(f_K)}^{-1} \left(q\right)\right)\,\mathrm{d}f_K\\
&=\int_{f_K} qP_{k-2}^{f_K}(w)\, \mathrm{d}f_K \hspace{2.2cm}\left(\text{by definition of }\Delta_{L^2(f_K)}^{-1}\right)\\
&=\int_{f_K} qw\,\mathrm{d}{f_K}\hspace{0.3cm}\left(\text{since }q\in\mathbb{P}_{k-2}(f_K)\text{ and definition of }P_{k-2}^{f_K}\right)\\
&= a^{f_K}(w,w).\hspace{3.7cm} \text{(equation \ref{auxpb} with }v=w)
\end{align*}
So since $w\in H_0^1(f_K)$, we have:
$$R(q)=0 \Leftrightarrow a^{f_K}(w,w)=0 \Leftrightarrow w=0 \text{ (}a^{f_K}\text{ coercive bilinear form) }\Leftrightarrow q=0.$$
The implication left-to-right of the last implication is obvious since any identically zero function has a (weak) Laplacian equal to $0$. The implication right-to-left comes from Lax-Milgram theorem and from the fact that the identically zero function in $f_K$ is zero on the boundary of $f_K$ and has a Laplacian equal to zero in $f_K$. Consequently, the map $R$ is injective. 
Moreover, since $v_h=0$ on $\partial f_K$, then $v_h\in H_0^1(f_K)$, and since $\Delta v_h \in \mathbb{P}_{k-2}(f_K)$, then:
$$P_{k-2}^{f_K}(v_h)=P_{k-2}^{f_K}\left(-\Delta_{L^2(f_K)}^{-1}
\left(-\Delta v_h\right)\right) = R\left(-\Delta v_h\right).$$
The first equality comes from the definition of $\Delta_{L^2(f_K)}^{-1}$, while the second one comes from the definition of $R$. Finally, we get
$$P_{k-2}^{f_K}(v_h)=0 \Rightarrow R\left(-\Delta v_h\right) = 0 \Rightarrow -\Delta v_h = 0.$$
Therefore, we have just proved that for all $f_K\in F_K$, then $\Delta v_h|_{f_K}=0$. Consequently, following what we have said before, $v_h|_{\partial K} \equiv 0$. 

We can now repeat the exact same proof with $v_h$ instead of $v_h|_{f_K}$, $a^K$ instead of $a^{f_K}$, and with $K$ instead of $f_K$, to show that we get $\Delta v_h = 0$ in $K$, knowing that $v_h = 0$ in $\partial K$ and assuming hypothesis $3.$ introduced at the beginning of the proof. Therefore, $v_h$ is identically zero in $K$, and thus the map that sends the degrees of freedom onto a function of $V^{K,k}$ is injective. Consequently, the degrees of freedom are unisolvent for $V^{K,k}$. 
\end{proof}

\begin{remark}
We emphasis the fact that the Laplace operator present in the definition of $V^{K,k}$ is here the most natural choice, but it could be replaced by any other second-order elliptic operator. The only mandatory and required properties on $V^{K,k}$ are that: 
\begin{itemize}
\item $\dim\left(V^{K,k}\right) = N^{K,k}$, 
\item $\mathbb{P}_{k}(K)\subset V^{K,k}$, 
\item $V^{K,k}$ is made of functions that are in polynomials of degree at most $k$ on each edge, 
\item the degrees of freedom $\mathcal{V}^{K,k}\cup\mathcal{E}^{K,k}\cup \mathcal{F}^{K,k}\cup\mathcal{P}^{K,k}$ are unisolvent for $V^{K,k}$.
\end{itemize}
\end{remark}

\subsubsection{Construction of the discretized space $V_h$}
Now, we have all the tools to build the discretized space $V_h$ on the whole space $\Omega$, that verifies all conditions of Theorem \ref{conditions}, as desired. So given a space decomposition $T_h$ into simple polygons, and for every integer $k\geq 1$, we define $V_h:=\bigcup_{K\in T_h} V^{K,k}$, with the right homogeneous Dirichlet boundary conditions imposed. More precisely, we define $V_h$ independently for the case $1$/$2$D or $3$D:\\

\noindent \underline{Case 1:} $d=1$ or $2$.
$$
V_h := \bigcup_{K\in T_h} V^{K,k} = \left\{v\in H^1_0(\Omega): v|_{\partial K} \in \mathbb{B}_k(\partial K) \text{ and } \Delta v|_K \in \mathbb{P}_{k-2}(K), \forall K\in T_h \right\}.
$$

\noindent \underline{Case 2:} $d=3$.
$$
V_h := \bigcup_{K\in T_h} V^{K,k} = \left\{v\in H^1_0(\Omega): v|_{\partial K} \in V^{\partial K, k} \text{ and } \Delta v|_K \in \mathbb{P}_{k-2}(K), \forall K\in T_h \right\}.
$$

\begin{proposition}
The dimension $N^\text{tot}$ of $V_h$ is equal to:
\begin{itemize}
\item $N^\text{tot} = N_v + N_p(k-1)$ if $d=1$;
\item $N^\text{tot} = N_v + N_e(k-1) + N_p\frac{k(k-1)}{2}$ if $d=2$;
\item $N^\text{tot} = N_v + N_e(k-1) + N_f\frac{k(k-1)}{2} + N_p\frac{k(k^2-1)}{6}$ if $d=3$, 
\end{itemize}
where $N_v$ is the total number of internal vertices, $N_e$ is the total number of internal edges, $N_f$ is the total number of internal faces, and $N_p$ is the total number of elements in $T_h$. 
\end{proposition}

\begin{proof}
This is deduced in a straightforward manner from Propositions \ref{dimVKkd12} and \ref{dimVKkd3}, keeping in mind that we impose homogeneous Dirichlet boundary conditions. This is why we only consider \emph{internal} vertices, edges and faces. 
\end{proof}

Then, the global degrees of freedom in $V_h$ are chosen in the same way as the local degrees of freedom. For all $v_h\in V_h$, $v_h$ is uniquely determined if we know the following values:
\begin{itemize}
\item the values of $v_h$ at each internal vertex of $T_h$. Let $\mathcal{V}$ be the set of such values;
\item if $d>1$, the values of $v_h$ at $k-1$ intermediate points on each internal edge of $T_h$. Let $\mathcal{E}$ be the set of such values;
\item if $d=3$, the moments up to order $k-2$ of $v_h$ on each internal face $f$ of $T_h$, that is $\frac{1}{|f|}\int_{f}m(\mathbf{x})v_h(\mathbf{x})\, \mathrm{d}\mathbf{x}$, for all $m\in \mathcal{M}_{k-2}(f)$, where as before, $$\mathcal{M}_{k-2}(D) = \left\{\left(\frac{\mathbf{x}-\mathbf{x}_D}{h_D}\right)^\mathbf{s} : |\mathbf{s}|\leq k-2\right\},$$ with $\mathbf{x}_D$ being the barycenter of the geometrical element $D$, $h_D$ its diameter, $|\mathbf{s}| = \sum_{i=1}^d s_i$ with $s_i\in \N_0$, and $\mathbf{x}^\mathbf{s}=\prod_{i=1}^d x_i^{s_i},$ $x_i$ being the $i$-th coordinate of $\mathbf{x}$. Let $\mathcal{F}$ be the set of such values.
\item the moments up to order $k-2$ of $v_h$ in every element $K\in T_h$, that is $\frac{1}{|K|}\int_Km(\mathbf{x})v_h(\mathbf{x})\, \mathrm{d}\mathbf{x}$, for all $m\in \mathcal{M}_{k-2}(K)$, where $\mathcal{M}_{k-2}(K)$ is defined in the previous point. Let $\mathcal{P}$ be the set of such values.
\end{itemize}
Finally, the homogeneous Dirichlet boundary conditions impose $v_h=0$ on the vertices, edges and faces belonging to $\partial \Omega$. In the exact same way as for Proposition \ref{chosenlocaldofnb}, it is straightforward to verify that the total number of chosen degrees of freedom (that is $\#\mathcal{P}+\#\mathcal{V}+\#\mathcal{E}+\#\mathcal{F}$) is indeed equal to the dimension $N^\text{tot}$ of $V_h$. Moreover, Proposition \ref{localunisolvent} implies that the degrees of freedom $\mathcal{P}\cup\mathcal{V}\cup\mathcal{E}\cup\mathcal{F}$ are unisolvent for $V_h$. \\
%If we assume that there exists $\gamma>0$ such that all the elements $K$ of $T_h$ are unions of a finite number of star-shaped domains $D$ with respect to a ball of radius at least $\gamma h_D$, then we can ensure the existence of an optimal local approximation in $\mathbb{P}_k(K)$ of any smooth function on $K$. The proof can be found in \textbf{TODO REFERENCE 9!!!}. \\

Consequently, since our aim is to be able to apply Theorem \ref{conditions}, it remains to build the symmetric bilinear form $a_h$ and the right-hand side element $f_h$ so that they approximate well the original problem and verify the properties needed. 

\subsubsection{Projection operator $\Pi^\nabla$} \label{projop}

To be able to build the symmetric bilinear form $a_h$, especially in $3$ dimensions, we need to introduce the projection operator $\Pi^\nabla:V^{K,k}\rightarrow \mathbb{P}_k(K)$, for all integer $k\geq 1$ and for all element $K\in T_h$. To be more precise, we should write this operator as $\Pi^\nabla_{K,k}$, but for a sake of simplicity, we will stick to the notation $\Pi^\nabla$. As before, we write $n_v^K$ the number of vertices of $K$. The operator is defined through the following orthogonality condition: for all $v\in V^{K,k}$, 
\begin{align}
&\int_K \nabla p\cdot\nabla\left(\Pi^\nabla v - v\right) \, \mathrm{d}K = 0, \text{ for all } p\in\mathbb{P}_k(K), \label{projopdef} \\
&\begin{cases}
\frac{1}{n_v^K}\sum_{i=1}^{n_v^K} v(v_i) = \frac{1}{n_v^K}\sum_{i=1}^{n_v^K} \Pi^\nabla v(v_i) \text{ if } k=1; \nonumber \\
\text{ or }\\ 
\frac{1}{|K|}\int_{K} v \, \mathrm{d}K = \frac{1}{|K|}\int_{K} \Pi^\nabla v \, \mathrm{d}K \text{ if } k\geq 2,
\end{cases}
\end{align}
where $\{v_i\}_{i=1}^{n_v^K}$ is the set of vertices of $K$. 

This projection operator is defined so that for every $v_h\in V^{K,k}$, we can compute $\Pi^\nabla v_h$ using only the degrees of freedom of $v_h$, defined in the previous section. The proof of this fact can be found in \cite{hitchhicker}.

Note that the projection operator $\Pi$ corresponds to the $a^K$-orthogonal projection on $\mathbb{P}_k(K)$. Moreover, if we take $v=q\in\mathbb{P}_k(K)\subset V^{K,k}$, then $\Pi^\nabla q = q$, so in particular, 
\begin{align*}
\|\nabla q\|^2_{L^2(K)} &= \int_K (\nabla q)^2 \, \mathrm{d}K \\
&\stackrel{(\ref{projopdef})}{=} \int_K \nabla q \cdot \nabla\left(\Pi^\nabla q\right) \,\mathrm{d}K \\
&\stackrel{(\ref{projopdef})}{=} \int_K \nabla\left(\Pi^\nabla q\right) \cdot \nabla\left(\Pi^\nabla q\right) \,\mathrm{d}K \\
&= \|\nabla\left(\Pi^\nabla q\right)\|^2_{L^2} .
\end{align*}
Consequently, $\Pi^\nabla q = q+C$, where $C$ is a constant, for all $q\in\mathbb{P}_k(K)$. But thanks to the second equation of (\ref{projopdef}), this constant is equal to $0$, so that $\Pi^\nabla q = q$. Let us see in the next section why this relation is useful. 

\subsubsection{Construction of the symmetric bilinear form $a_h$}
Let us recall that for all $K\in T_h$, for all $v,w\in V=H_0^1(\Omega)$, $a^K(w,v)= \int_K \nabla w \cdot \nabla v \, \mathrm{d}K$ and $a(w,v)=\sum_{K\in T_h} a^K(w,v)$. Now, for any element $K$ of $T_h$, if we fix an integer $k\geq 1$, then for all $v\in V^{K,k}$ and for all $p\in \mathbb{P}_k(K)$, 
\begin{equation}\label{comput}
a^K(p,v)= \int_K \nabla p \cdot \nabla v \, \mathrm{d}K = - \int_K \Delta p v \, \mathrm{d}K +  \int_{\partial K} \frac{\partial p}{\partial n} v \, \mathrm{d}s,
\end{equation}
thanks to an integration by parts.
Since $\Delta p\in \mathbb{P}_{k-2}(K)$, $\Delta p$ can be written with respect to the basis of $\mathbb{P}_{k-2}(K)$ made by $\mathcal{M}_{k-2}(K)$, so that it is a linear combination of moments. Then, thanks to the chosen degrees of freedom, the first integral of the right hand side of equation (\ref{comput}) can be exactly computed. 

Moreover, for every $e\in\partial K$, $\frac{\partial p}{\partial n}\in \mathbb{P}_{k-1}(e)$, and if $d=1$ or $d=2$, $v\in \mathbb{P}_k(e)$. So $\frac{\partial p}{\partial n}v\in \mathbb{P}_{2k-1}(e)$. In the case $d=1$, $\partial K$ is composed of $2$ points, i.e. vertices and edges are equal. So since we know $v_h$ at each internal vertex of $T_h$, then the second integral of the right hand side of equation (\ref{comput}) can be exactly computed. In the case $d=2$, $\partial K$ is composed of edges, and we know $v$ on $k$ points on each edge $e\in\partial K$ thanks to the chosen degrees of freedom. So if we choose an appropriate quadrature formula (Gauss-Legendre-Lobatto is fine), we can also compute exactly the second integral of the right hand side of equation (\ref{comput}). A good choice of degrees of freedom on each edge of $T_h$ would thus be the $k-1$ internal Gauss-Legendre-Lobatto points, plus the vertices. 

Finally, if $d=3$, $\partial K$ is composed of faces. Unlike the case $d=1$ or $2$, on any face, $v$ does not necessarily belong to $\mathbb{P}_k(e)$. Consequently, to be able to compute exactly $\int_{\partial K} \frac{\partial p}{\partial n} v \, \mathrm{d}s$, we need to  slightly modify the definition of $V^{K,k}$ for $d=3$. Recall that $V^{K,k}$ has been defined from $V^{f_K,k}$ for all $f_K\in F_K$, so that we now make the following choice:
\begin{align*}
V^{f_K,k} := &\Bigg\{v\in H^1(f_K) : v|_{\partial f_K} \in \mathbb{B}_k(\partial f_K), \Delta v\in \mathbb{P}_{k}(f_K), \\
&\left. \int_{f_K} v(\mathbf{x}) m(\mathbf{x}) \, \mathrm{d}\mathbf{x} = \int_{f_K} \Pi^\nabla v(\mathbf{x}) m(\mathbf{x})\mathrm{d}\mathbf{x}, \forall m\in\mathcal{M}_{k-1,k}(f_K) \right\},
\end{align*}
where we recall again that $$\mathcal{M}_{k-1,k}(f_K) = \left\{\left(\frac{\mathbf{x}- \mathbf{x}_{f_K}}{h_{f_K}}\right)^\mathbf{s} : |\mathbf{s}|= k-1, k\right\},$$ with $\mathbf{x}_{f_K}$ being the barycenter of the face $f_K$, $h_{f_K}$ its diameter, $|\mathbf{s}| = \sum_{i=1}^d s_i$ with $s_i\in \N_0$, and $\mathbf{x}^\mathbf{s}=\prod_{i=1}^d x_i^{s_i},$ $x_i$ being the $i$-th coordinate of $\mathbf{x}$. Note that here, the projection operator $\Pi^\nabla$ projects functions belonging to $V^{f_K,k}$ into $\mathbb{P}_k(f_K)$, for any $f_K\in F_K$. 
Unlike the previous definition, the Laplacian is not required to be in $\mathbb{P}_{k-2}$ but in $\mathbb{P}_k$, and extra conditions on the moments of the function are required.

Now we have to check that the dimension of $V^{f_K,k}$ is still the same as before. 
To know the dimension of $V^{f_K,k}$, we first look more specifically into the last two conditions. $\Delta v\in\mathbb{P}_k(f_K)$ gives us $\dim\big(\mathbb{P}_k(f_K)\big)$ conditions. Moreover, we know that the moments on $f_K$ up to some order $\tilde{k}$ form a basis of $\mathbb{P}_{\tilde{k}}(f_K)$. Consequently, imposing the last condition of the definition of $V^{f_K,k}$ for all $m\in\mathcal{M}_{k-1,k}(f_K)$ and imposing $\Delta v\in\mathbb{P}_k(f_K)$ is equivalent to being left with $\dim\big(\mathbb{P}_{k-2}(f_K)\big)$ degrees of freedom. Therefore, 
$$\dim\left(V^{f_K,k}\right) = \dim\big(\mathbb{B}_k\left(\partial f_K\right)\big)+\dim\big(\mathbb{P}_{k-2}(f_K)\big)= n_v^{f_K} + n_e^{f_K}(k-1) + \frac{k(k-1)}{2}.$$
For a more rigorous proof of this, and to prove the existence of functions in $V^{f_K,k}$, please refer to \cite{existenceW}. Note moreover that $V^{K,k}$ built from $V^{f_K,k}$ can still be described with the same degrees of freedom as previously, and it still contains all the polynomials of degree $k$. 

Thanks to this new definition which will always be used from now on, one can prove that in the $3$D case, the second integral of the right hand side of equation (\ref{comput}) can be exactly computed from the chosen degrees of freedom. Indeed, since $\frac{\partial p}{\partial n}\in \mathbb{P}_{k-1}(f)$, for all face $f\in \partial K$, and since $\mathcal{M}_{k-1}(f)$ is a basis for $\mathbb{P}_{k-1}(f)$, then $\frac{\partial p}{\partial n}$ can be decomposed into a linear combination of moments of order at most $k-1$. Then, thanks to the last property of $V^{f,k}$, we know that for all $m\in\mathcal{M}_{k-1}(f)$, $\int_{f} v(\mathbf{x}) m(\mathbf{x}) \, \mathrm{d}\mathbf{x} = \int_{f} \Pi^\nabla v(\mathbf{x}) m(\mathbf{x})\mathrm{d}\mathbf{x}$. Any considered moment has degree at most $k-1$, and $\Pi^\nabla v$ is by definition a polynomial of degree at most $k$. So $\int_{\partial K} \frac{\partial p}{\partial n}v \, \mathrm{d}s$ can be written as a linear combination of integrals of polynomials of degree at most $2k-1$. Since we know from section \ref{projop} that we can exactly compute $\Pi^\nabla v$  using only the degrees of freedom of $v$, then if we choose an appropriate quadrature formula (as Gauss-Legendre-Lobatto), we can compute exactly the second integral of the right hand side of equation (\ref{comput}). 

Consequently, with the chosen degrees of freedom and for $d=1,2$ or $3$, it is possible to compute exactly the value of $a^K(p,v)$ for any $K\in T_h$, $p\in \mathbb{P}_k(K)$ and $v\in V^{K,k}$. This result will be used to build a computable symmetric bilinear form $a_h$ that satisfies the conditions of Theorem \ref{conditions}. \\

Let $k\geq 1$ and $K\in T_h$. Now, we want to build $a_h$ such that it verifies the conditions of Theorem \ref{conditions}, in particular the $k$-consistency and the stability properties that we recall here: 
\begin{itemize}
\item($k$-consistency) for all $p\in \mathbb{P}_k(K)$, for all $v_h\in V_h|_K$, the relation $a_h^K(p, v_h) = a^K(p, v_h)$ is verified;
\item (stability) there exist $\alpha^-, \alpha^+ \geq 0$ two constants independent of $h$ and of $K$ such that for all $v_h\in V_h|_K$,
$$ \alpha^-a^K(v_h,v_h)\leq a_h^K(v_h,v_h)\leq \alpha^+a^K(v_h,v_h).$$
\end{itemize}

In section \ref{projop}, we have seen that for all $q\in \mathbb{P}_k(K)$, $\Pi^\nabla q = q$, so that a natural choice to satisfy the $k$-consistency would be to take for all $u,v\in V^{K,k}, a_h^K(u,v) = a^K(\Pi^\nabla u, \Pi^\nabla v)$. However, in this case, stability cannot always be verified, so that an other term needs to be added, as explained in the following theorem. 

\begin{theorem} \label{introSK}
Let $S^K$  be a symmetric bilinear form such that there exist two constants $c_0, c_1>0$ independent from $K$ and $h$ that verify
\begin{equation} \label{condSK}
c_0 a^K(v,v) \leq S^K(v,v) \leq c_1a^K(v,v), \forall v\in V^{K,k} \text{ with } \Pi^\nabla v = 0.
\end{equation}
If for all $u,v\in V^{K,k}$, $a_h^K$ is defined by $$a_h^K(u,v):= a^K(\Pi^\nabla u, \Pi^\nabla v) + S^K(u-\Pi^\nabla u, v-\Pi^\nabla v),$$ then $a_h^K$ is a bilinear form that satisfies both the $k$-consistency and the stability properties.
\end{theorem}

\begin{remark}
Note that the definition of $a_h^K$ in the previous theorem is inspired by the Pythagoras theorem on $a^K$, applied as follows. Since $\Pi^\nabla$ is the $a^K$-orthogonal projection of $V^{K,k}$ on $\mathbb{P}_k(K)$, we have for all $u,v\in V^{K,k}$,
$$
a^K(u,v) = a^K(\Pi^\nabla u, \Pi^\nabla v) + a^K(u-\Pi^\nabla u, v-\Pi^\nabla v).
$$
\end{remark}

\begin{proof}
First, $a_h^K$ is a symmetric bilinear form since $a^K$ and $S^K$ are, and since $\Pi^\nabla$ is linear too. Then, for all $p\in \mathbb{P}_k(K)$, we know that $p=\Pi^\nabla p$, so $S^K(p-\Pi^\nabla p, v-\Pi^\nabla v)=0,$ for all $v\in V^{K,k}$. Consequently, by definition of $a_h^K$ and of $\Pi^\nabla$, $$a_h^K(p,v) = a^K(\Pi^\nabla p, \Pi^\nabla v) = a^K(p, \Pi^\nabla v) = a^K(p,v),$$
that is the $k$-consistency property is verified. 

Finally, for all $v\in V^{K,k}$, $\Pi^\nabla\left(\Pi^\nabla v - v\right) = \Pi^\nabla v - \Pi^\nabla v = 0$ by linearity of $a^K$ and since $\Pi^\nabla v\in \mathbb{P}_k(K)$. So thanks to the definition of $a_h^K$ and to relation (\ref{condSK}), 
\begin{align*}
a_h^K(v,v)&\leq a^K(\Pi^\nabla v, \Pi^\nabla v) + c_1a^K(v-\Pi^\nabla v, v-\Pi^\nabla v)\\
&\leq \max\{1,c_1\}\big(a^K(\Pi^\nabla v, \Pi^\nabla v) + a^K(v-\Pi^\nabla v, v-\Pi^\nabla v)\big).
\end{align*}
Very similarly, for all $v\in V^{K,k}$, 
\begin{align*}
a_h^K(v,v)&\leq a^K(\Pi^\nabla v, \Pi^\nabla v) + c_0a^K(v-\Pi^\nabla v, v-\Pi^\nabla v)\\
&\geq \min\{1,c_0\}\big(a^K(\Pi^\nabla v, \Pi^\nabla v) + a^K(v-\Pi^\nabla v, v-\Pi^\nabla v)\big).
\end{align*}
Consequently, $a_h^K$ also verifies the stability property.
\end{proof}

It remains to find a suitable symmetric positive definite bilinear form $S^K$. In order to verify relation (\ref{condSK}), we need $S^K$ to scale like $a^K$ on $\Ker(\Pi^\nabla)$. We recall that $N^{K,k} = \dim(V^{K,k})$. Let $\{\phi_i\}_{i=1}^{N^{K,k}}$ be the Lagrangian basis of $V^{K,k}$, that is dof$_i(\phi_i)=\delta_{ij}, \forall i,j=1,\ldots,N^{K,k}$, where dof$_i:V^{K,k}\rightarrow \R$ maps each function of $V^{K,k}$ to its $i$-th degree of freedom. Before defining $S^K$, we need two lemmas.

\begin{lemma}\label{lemma1}
All the degrees of freedom scale as $1$, that is they are invariant under rescaling of the elements of the mesh. 
\end{lemma}
\begin{proof}
Let $\hat{K}$ be an element of $T_h$ and consider the change of variable $\mathbf{x}=h\hat{\mathbf{x}}$ that maps the element $\hat{K}$ onto an element $K$. For each basis function $\hat{\phi}_i$, $i=1,\ldots,N^{\hat{K},k}$, we define $\phi_i$ on $K$ as $\phi_i(\mathbf{x}) = \phi_i(h\mathbf{\hat{x}}) := \hat{\phi}_i(\hat{\mathbf{x}}) = \hat{\phi}_i\left(\frac{\mathbf{x}}{h}\right)$, and $N^{K,k}=N^{\hat{K},k}$. If $\hat{\phi}_i$ is a basis function corresponding to a vertex or an edge-point degree of freedom, then dof$_i(\phi_i)=$dof$_i(\hat{\phi}_i) = 1$ since the change of variable maps vertices of $\hat{K}$ to vertices of $K$ and edge-points of $\hat{K}$ to edge-points of $K$. If instead, $\hat{\phi}_i$ is a basis function corresponding to a moment degree of freedom, then taking $\hat{D}=\hat{f}$ any face of $\hat{K}$ (if $d=3$) and $m=2$, or $\hat{D}=\hat{K}$ and $m=3$, and $D$ the image of $\hat{D}$ by the change of variables, then: for all $|\mathbf{s}|\leq k-2$, 
\begin{align*}
1&=\frac{1}{|\hat{D}|}\int_{\hat{D}} \hat{\phi}_i(\mathbf{\hat{x}}) \left(\frac{\mathbf{\hat{x}}-\mathbf{\hat{x}}_{\hat{D}}}{h_{\hat{D}}}\right)^\mathbf{s}\,\mathrm{d}\mathbf{\hat{x}} \\
&= \frac{h^m}{|D|}\int_D \phi_i(\mathbf{x})\left(\frac{ h\left(\mathbf{x}-\mathbf{x}_D\right)}{hh_D}\right)^\mathbf{s}\frac{1}{h^m}\,\mathrm{d}\mathbf{x} \\
&= \frac{1}{|D|} \int_D \phi_i(\mathbf{x})\left(\frac{\mathbf{x}-\mathbf{x}_D}{h_D}\right)^\mathbf{s}\,\mathrm{d}\mathbf{x}.
\end{align*} 
\end{proof}

\begin{lemma} \label{lemma2}
Given $h>0$ the size of a mesh decomposition $T_h$, $K$ an element of $T_h$ and $k\geq 1$ an integer, then $h^{2-d}a^K(\phi_i,\phi_i)$ scales like $1$ for all $i=1,\ldots,N^{K,k}$. That is, $h^{2-d}a^K(\phi_i,\phi_i)$ is independent from $K$ and $h$, or equivalently, it is invariant under rescaling the elements of the mesh. 
\end{lemma}
\begin{proof}
Let $\hat{K}$ be an element of $T_{\hat{H}}$ and consider the change of variables $\mathbf{x}=h\hat{\mathbf{x}}$ that maps the element $\hat{K}$ onto an element $K$. More generally, this change of variables maps any element of the mesh $T_{\hat{H}}$ onto an element of the mesh $T_H$, where $H=h\hat{H}$. As in the previous lemma, for each basis function $\hat{\phi}_i$, $i=1,\ldots,N^{\hat{K},k}$, we define $\phi_i$ on $K$ as $\phi_i(\mathbf{x}) = \phi_i(h\mathbf{\hat{x}}) := \hat{\phi}_i(\hat{\mathbf{x}}) = \hat{\phi}_i\left(\frac{\mathbf{x}}{h}\right)$, and $N^{K,k}=N^{\hat{K},k}$. Then for all $i=1,\ldots,N^{K,k}$, 
\begin{align*}
\hat{H}^{2-d}a^{\hat{K}}(\hat{\phi_i,\phi_i}) &= \hat{H}^{2-d}\int_{\hat{K}} |\nabla \hat{\phi_i}|^2 \, \mathrm{d}\hat{K} \\
&= \frac{H^{2-d}}{h^{2-d}} \int_K |\nabla \phi_i h|^2 \frac{1}{h^d} \, \mathrm{d}K\\
&= H^{2-d} \int_K |\nabla \phi_i|^2 \, \mathrm{d}K.
\end{align*}
\end{proof}

Thanks to the two previous lemma, we can now choose correctly the symmetric bilinear form $S^K$ introduced in Theorem \ref{introSK}. 
\begin{proposition}
$S^K:V^{K,k}\times V^{K,k}\rightarrow \R$ defined by $S^K(u,v) = h^{d-2}\sum_{r=1}^{N^{K,k}} \text{\emph{dof}}_r(u)\text{\emph{dof}}_r(v)$, for all $u,v\in V^{K,k}$, is a symmetric bilinear form that verifies condition (\ref{condSK}) of Theorem \ref{introSK}. 
\end{proposition}
\begin{proof}
It is straightforward to see that $S^K$ is symmetric. Moreover, since the functions dof$_i$ are linear for all $i=1,\ldots,N^{K,k}$, then $S^K$ is bilinear. It remains to prove condition (\ref{condSK}). Let $v\in V^{K,k}\cap \Ker(\Pi^\nabla)$. Since $\{\phi_i\}_{i=1}^{N^{K,k}}$ is a basis of $V^{K,k}$, then there exist $v_1,\ldots,v_{N^{K,k}}\in \R$ such that $v=\sum_{i=1}^{N^{K,k}} v_i\phi_i$. Consequently, 
$$S^K(v,v) = h^{d-2}\sum_{r=1}^{N^{K,k}} \big(\text{dof}_r\left(v\right)\big)^2 = h^{d-2}\sum_{r=1}^{N^{K,k}} v_r^2.$$
Moreover, 
\begin{align*}
a^K(v,v) &= \sum_{r=1}^{N^{K,k}} v_r^2 a^K(\phi_r,\phi_r)\\
&\leq \max_r\Big\{a^K(\phi_r, \phi_r)\Big\} \sum_{r=1}^{N^{K,k}} v_r^2\\
&= \max_r\Big\{h^{2-d}a^K(\phi_r, \phi_r)\Big\} h^{d-2} \sum_{r=1}^{N^{K,k}} v_r^2;\\
a^K(v,v) &= \sum_{r=1}^{N^{K,k}} v_r^2 a^K(\phi_r,\phi_r)\\
&\geq \min_r\Big\{a^K(\phi_r, \phi_r)\Big\} \sum_{r=1}^{N^{K,k}} v_r^2\\
&= \min_r\Big\{h^{2-d}a^K(\phi_r, \phi_r)\Big\} h^{d-2} \sum_{r=1}^{N^{K,k}} v_r^2.
\end{align*}
Since we know from \ref{lemma2} that $h^{2-d}a^K(\phi_r,\phi_r)$ is independent from $h$ and $K$ for all $r=1,\ldots,N^{K,k}$, then $\frac{1}{c_1}:=\min_r\Big\{h^{2-d}a^K(\phi_r, \phi_r)\Big\}$ and $\frac{1}{c_0}:=\max_r\Big\{h^{2-d}a^K(\phi_r, \phi_r)\Big\}$ are two constants independent from $h$, $K$ and $r$. Moreover, $h^{d-2}\sum_{r=1}^{N^{K,k}}v_r^2=S^K(v,v)$, so that:
$$ c_0a^K(v,v)\leq S^K(v,v) \leq c_1a^K(v,v),$$
and thus $S^K$ verifies property (\ref{condSK}).
\end{proof}

To sum up, we have been able to define $a_h^K$ by $$a_h^K(u,v):=a^K(\Pi^\nabla u, \Pi^\nabla v) + S^K(u-\Pi^\nabla u, v-\Pi^\nabla v),$$ 
for all $u,v\in V^{K,k}$, so that it verifies all the conditions of Theorem \ref{conditions}. The last remaining part to apply this theorem is to build the right-hand side element $f_h$.

\subsubsection{Construction of the right-hand side $f_h$}
In this part is presented the choice made to approximate $F$ from the weak formulation of the differential problem (\ref{weak}). Since $F$ is a functional on the Hilbert space $V=H_0^1(\Omega)$ (the considered norm in $V$ is the semi-norm of $H^1(\Omega)$), we know by Riesz representation theorem that there is a unique element $x_f\in V$ such that for all $v\in V$, $F(v) = \int_\Omega \nabla v\cdot\nabla x_f\, \mathrm{d}\Omega$. We separate the case $k=1$ to the case $k\geq 2$ for technical reasons that will become obvious in the following.\\

\noindent \underline{Case 1:} $k\geq 2$. \\
$f_h$ is defined as the $L^2(K)$-projection of $x_f$ onto the space $\mathbb{P}_{k-2}(K)$, on each $K\in T_h$. Let us write it $f_h := P_{k-2}^Kx_f$, for all $K\in T_h$. In this way, we have: for all $v_h\in V_h$, 
\begin{equation} \label{fhk2}
\langle f_h, v_h \rangle = \sum_{K\in T_h} \int_K f_h v_h \, \mathrm{d}K := \sum_{K\in T_h}\int_K \left(P_{k-2}^Kx_f\right)v_h\, \mathrm{d}K
\end{equation}
Since for all $K\in T_h$, $P_{k-2}^Kx_f\in \mathbb{P}_{k-2}(K)$ and $\mathcal{M}_{k-2}(K)$ is a basis of $\mathbb{P}_{k-2}(K)$, then $P_{k-2}^Kx_f$ can be written as a linear combination of the elements of $\mathcal{M}_{k-2}(K)$. Plugging it into equation (\ref{fhk2}) gives us a linear combination of moments of $v_h$ of order at most $k-2$ in every element $K$ of $T_h$. These values are known since they exactly corresponds to some of the degrees of freedom of $v_h$. Consequently, this is a quantity that can be computed directly from the degrees of freedom. \\

\noindent \underline{Case 2:} $k=1$. \\
$f_h$ is a piecewise constant function defined as follows: for all $v_h\in V_h$, 
$$\langle f_h, v_h \rangle := \sum_{K\in T_h} \int_K P_0^Kx_f\, \bar{v_h} \,\mathrm{d}K = \sum_{K\in T_h}P_0^Kx_f\,\bar{v_h},$$
where $\bar{v_h} = \frac{1}{n_v^K}\sum_{i=1}^{n_v^K}v_h(V_i)$ with $\{V_i\}_{i=1}^{n_v^K}$ the set of vertices of $K$. 
Since the value of $v_h$ at the vertices of every element are amongst the degrees of freedom of $v_h$, then this is a computable quantity from the degrees of freedom only. \\

Therefore, we have now in hand all the tools to apply Theorem \ref{conditions}. In the next section, we will state results about approximation and interpolation errors, to make more precise the estimation of the $H^1$-error of the solution present in the theorem. 

\subsubsection{Approximation and projection errors} \label{errors}
Only results without proofs will be presented in this section. However, references where to find the proofs will be given. Moreover, on chapter \ref{results} of this report will be reported the numerical errors found on some particular cases, together with some convergence numerical analysis. 

\begin{theorem}[Projection error]
Assume that there exists $\gamma>0$ such that for all $h$, each element $K$ in $T_h$ is a union of a finite number of star-shaped domains with respect to any point of a ball of radius greater then $\gamma h_K$. Then there exists a constant $C = C(k,\gamma)$ such that for every $s$ with $1\leq s\leq k+1$ and for every $w\in H^s(K)$, there is $w_\pi\in\mathbb{P}_k(K)$ such that 
$$
\|w-w_\pi\|_{L^2(K)} + h_K|w-w_\pi|_{H^1(K)} \leq Ch_K^s|w|_{H^s(K)}.
$$
\end{theorem}
\begin{proof}
The proof can be found in \cite{brennerscott} by S.C. Brenner and R.L. Scott.
\end{proof}

\begin{theorem}[Interpolation error]
Assume that there exists $\gamma>0$ such that for all $h$, each element $K$ in $T_h$ is a union of a finite number of star-shaped domains with respect to any point of a ball of radius greater then $\gamma h_K$. Then there exists a constant $C=C(k,\gamma)$ such that for every $s$ with $2\leq s\leq k+1$, for every $h$, for all $K\in T_h$ and for all $w\in H^s(K)$, there exists $w_I\in V^{K,k}$ such that
$$
\|w-w_I\|_{L^2(K)} + h_K|w-w_I|_{H^1(K)} \leq Ch_K^s|w|_{H^s(K)}.
$$
\end{theorem}
\begin{proof}
The proof can also be found in \cite{brennerscott} by S.C. Brenner and R.L. Scott.
\end{proof}

\begin{theorem}[Approximation of the right-hand side]
If $k=1$, for all $v_h\in V_h$, there exists a constant $C$ such that
$$
\langle f_h, v_h\rangle - F(v_h) \leq Ch\left(\sum_{K\in T_h} |f|^2_{H^1(K)}\right)^{\frac{1}{2}}|v_h|_{H^1(K)}.
$$
Now, if $k\geq 2$, for all $v_h\in V_h$, there exists a constant $C$ such that 
$$
\langle f_h, v_h\rangle - F(v_h) \leq Ch^k\left(\sum_{K\in T_h} |f|^2_{H^{k-1}(K)}\right)^{\frac{1}{2}}|v_h|_{H^1(K)}.
$$
\end{theorem}
\begin{proof}
The proof can be found in \cite{basicpples} by L. Beirão da Veiga, F. Brezzi, A. Cangiani, G. Manzini, L.D. Marini and A.Russo. 
\end{proof}
