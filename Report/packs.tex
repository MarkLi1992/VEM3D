%PACKAGES--------------------------------
%Langue et encodage
\usepackage[english]{babel}
\usepackage[utf8]{inputenc}
\usepackage[T1]{fontenc}
\usepackage{lmodern}
%Gestion de mise en page
\usepackage{supertabular}
%\usepackage{fullpage}
\usepackage{here}% permet, avec [H] de mettre l'image ICI!!! (et pas au petit bonheur la chance LaTeXien)
\usepackage[perpage,symbol]{footmisc} %remet à zéro le compteur des notes de bas de page après chaque page
\usepackage[hmargin={3cm,3cm}, vmargin={3cm,3cm}, dvips]{geometry}
\usepackage{url}
\usepackage{listings}
\lstdefinestyle{customc}{
  belowcaptionskip=1\baselineskip,
  language=C,
  showstringspaces=false,
  basicstyle=\footnotesize\ttfamily,
  keywordstyle=\bfseries\color{green!40!black},
  commentstyle=\itshape\color{purple!40!black},
  identifierstyle=\color{blue},
  stringstyle=\color{orange},
}
\lstset{escapechar=@,style=customc}

%\pagestyle{plain}
%\parindent=0cm	%définit la longueur du retrait
%Mathématiques
\usepackage{amsmath} %Formules
\usepackage{amssymb} %Formules
\usepackage{amsthm} %Environnements
%\usepackage{relsize,exscale} %changement taille symboles mathématiques
%\usepackage{amsfonts}
\usepackage{enumerate}
\usepackage{esint}
\usepackage{graphics}
\usepackage{graphicx}
\usepackage{tikz}
%\usepackage{mathrsfs}
%\usepackage{mathtools}
%\usepackage{multirow}
%\usepackage[all]{xy} %xy-pic
%style captions
%\usepackage[font=small,labelfont=bf]{caption}

% pour changer le nom des captions des figures
\usepackage{ccaption}
\addto\captionsfrench{\def\figurename{{Diagramme}}}

%ENVIRONNEMENTS--------------------------
%Définition des styles
\newtheoremstyle{droit}
{}%Space above
{}%Space below
{\upshape}%Body font
{}%Indent amount (empty = no indent, \parindent = para indent)
{\bfseries}%theoreme head font
{}%Punctuation after theoreme head
{ }%{\newline}%Space after theoreme head: " " = normal interword space; \newline = linebreak
{}%theoreme head spec (can be left empty, meaning `normal')
\newtheoremstyle{italique}
{}%Space above
{}%Space below
{\itshape}%Body font
{}%Indent amount (empty = no indent, \parindent = para indent)
{\bfseries}%theoreme head font
{}%Punctuation after theoreme head
{ }%{\newline}%Space after theoreme head: " " = normal interword space; \newline = linebreak
{}%theoreme head spec (can be left empty, meaning `normal')
%Théorèmes
\theoremstyle{italique}
\newtheorem{theorem}{Theorem}[section]
\newtheorem{proposition}[theorem]{Proposition}
\newtheorem{lemma}[theorem]{Lemma}
\newtheorem{corollary}[theorem]{Corollary}
\newtheorem{axiom}[theorem]{Axiom}
\theoremstyle{droit}
\newtheorem{example}[theorem]{Example}
\newtheorem{remark}[theorem]{Remark}
\newtheorem{note}[theorem]{Note}
\newtheorem{definition}[theorem]{Definition}
\newtheorem{notation}[theorem]{Notation}
\newtheorem{terminology}[theorem]{Terminology}
\newtheorem{question}[theorem]{Question}
%Autres
\renewcommand{\thefootnote}{\fnsymbol{footnote}} %dans les notes de bas de page, remplace les chiffres par un symbole
\newenvironment{indice} %comme une liste, mais avec (i), (ii), (iii), (iv)
{\begin{enumerate}[\textup{(}i\textup{)}] }
{\end{enumerate}}
\newenvironment{step}
{\begin{enumerate}[\textup{Step}1\textup{ :}] }
{\end{enumerate}}
\newenvironment{case}
{\begin{enumerate}[\underline{\textup{Case} 1\textup{ :}}] }
{\end{enumerate}}

%COMMANDES-------------------------------
\newcommand{\N}{\mathbb{N}}
\newcommand{\Z}{\mathbb{Z}}
\newcommand{\Q}{\mathbb{Q}}
\newcommand{\R}{\mathbb{R}}
\newcommand{\C}{\mathbb{C}}
%\newcommand{\L}{\mathcal{L}}
\newcommand{\Ker}{{\rm Ker}}
\newcommand{\Tr}{{\rm Tr}}
\newcommand{\im}{{\rm Im}}
\newcommand{\ch}{{\rm ch}}
\DeclareMathOperator{\spn}{span}
\newcommand{\dans}{\longrightarrow}
\newcommand{\ddans}{\rightrightarrows}
%\newcommand{\tndans}{\Rightarrow}
\newcommand{\Bdot}{\bullet}
\newcommand{\noi}{\noindent}

