\newpage
\section*{Notations}

In this paragraph, we are going to describe the notation that we will use in this document. \\

\begin{tabular}{ll}
   $\#S$ & Cardinality of a set $S$ / Order of a group $S$ \\
   $M_n(K)$ & Ring of all $n \times n$ matrices with coefficients in a field $K$ \\
   $GL_n(K)$ & Group of all $n \times n$ non singular matrices with coefficients \\ & in a field $K$\\
   $GL(V)$ & Group of all linear transformations from $V$ to $V$, with $V$ \\ & a vector space \\
   $T_n(K)$ & Set of all $n \times n$ triangular matrices with coefficients in a \\&field $K$ \\
   $U_n(K)$ & Set of all $n \times n$ triangular matrices with coefficients in a \\&field $K$ and with $1$'s in the diagonal\\
   $(I_n)^R$ & Identity matrix in $R$ of dimension $n \times n$; we also write $I_n$ \\& if there is no ambiguity on $R$, or even $I$ of $1$ if the dimension \\& is implicit \\
   $\Tr(A)$ & Trace of the matrix $A$ \\
   $A^T$ & Transpose of the matrix $A$ \\
   $A_{ij}$ & Entry $(i,j)$ -- $i^{\text{th}}$ row and $j^{\text{th}}$ column -- of the matrix $A$\\
%  $C_n$ & Cyclic group of order $n$ \\
   $\Ker(f)$ & Kernel of a function $f$ \\
   $\im(f)$ & Image of a function $f$ \\
   $\N$ & $\{1,2,\ldots\}$, the set of positive integers \\
   $\N_0$ & $\{0,1,2,\ldots\}$ the set of non negative integers\\
   $\Z, \Q, \R, \C$ & Sets of integers, rational numbers, real numbers and complex \\& numbers \\
   $\C^{\ast}$ & Set of complex numbers without $0$ \\
   $A:=B$ & $A$ is by definition equal to $B$\\
   $\dim_K(V)$ & Dimension of the $K$-vector space $V$; we also write $\dim(V)$ \\& if there is no ambiguity on the field $K$\\
   ${\rm ch}(K)$ & Characteristic of a field or a division ring $K$\\
%   $G/H$ & Group (or vector space) quotient of $G$ by $H$\\
%   $\deg(f)$ & Degree of the polynomial $f$\\ \\
\end{tabular}\\

For any group's law, we will generally adopt the multiplicative notation. However, if the group is abelian, we will adopt the additive notation. \\