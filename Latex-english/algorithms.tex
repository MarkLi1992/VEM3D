\newpage
\section{Algorithms}
In this chapter a theoretical description of the algorithms used is given, to determine some specific properties of the mesh elements. Indeed, one of the principal characteristics of VEM is the possibility to consider meshes with any type of polygons/polyhedrons, even possibly concave. 

Since it has not been possible to find one single article speaking about all of the encountered problems, the principal algorithms used are presented here. Most of them have been found online (see \cite{bourke1988calculating}\cite{newell}\cite{rayCasting}), but we have modified some to adapt them to the problem.

\subsection{Polygons}\label{polygons}
In our implementation, a polygon is defined as an \textit{ordered} set of points. It is necessary to compute:
\begin{itemize}
\item its area,
\item an oriented normal vector,
\item its barycenter.
\end{itemize}
The implementation of each of those characteristics is different if the polygon belongs to a $2$D or to a $3$D space.

\subsubsection{Area in $2$D}\label{area2d}
Let $n$ be the number of vertices of a given polygon, and $(x_i,y_i)$ the coordinates of the $i$-th vertex, with $1\leq i\leq n$. To compute the area $A$ of a polygon, we have used the following Gauss-Green formula \cite{bourke1988calculating}:
$$ A=\frac{1}{2} \bigg | \sum_{i=1}^n (x_iy_{i+1}-x_{i+1}y_i) \bigg
|,$$
where the following convention is used: $x_{n+1}=x_1,y_{n+1}=y_1$.

\subsubsection{Area in $3$D}\label{area3d}
For the $3$D case, we have used a similar algorithm. Let $(O;\mathbf{e_x}, \mathbf{e_y}, \mathbf{e_z})$ be an orthonormal Cartesian coordinate system of the space, and let $V=\lbrace V_i : 1\leq i\leq n\rbrace$ be the set of vertices of a given $n$-vertex polygon. For $i=1,\ldots,n$, let $(v_{1i},v_{2i},v_{3i})$ be the the coordinates of $V_i$. We also write {\color{red}consider?} $\mathbf{n}$ a normal vector to the polygon, with coordinates $(n_1,n_2,n_3)$. The algorithm is as follows:
\begin{enumerate}
\item \label{pt1area3d} Project the polygon on one of the three principal plane $(O;\mathbf{e_x},\mathbf{e_y})$, $(O;\mathbf{e_x},\mathbf{e_z})$ or $(O;\mathbf{e_y},\mathbf{e_z})$ in the following way: set to $0$ the component $v_{ji}$ of each vertex $i$ such that $j=\text{arg} \max_{1\leq k \leq 3} \|n_k\|$. In this way, the polygon is equivalent to a polygon embedded in a $2$D space instead of a $3$D one, and setting to $0$ such component of all the vertices avoids the case of a polygon that is already parallel to one of the three principal planes. That is, we avoid the case of a degenerated polygon by choosing in a reasonable way on which plane we project it.  
\item Compute the area of the projected polygon with Gauss-Green formula (see paragraph \ref{area2d}). 
\item Divide the area of the projected polygon by the following scale factor: 
$$ \frac{\max_{1\leq i \leq 3}(n_i)}{\|n\|}. $$
The results gives the area of the original polygon embedded in a $3$D space (see \textbf{REFERENCE} {\color{red} no reference}).
\end{enumerate}

\subsubsection{Oriented normal vector}\label{onv}
Since in our definition of a polygon, the vertices are ordered, the orientation of the normal is obtained thanks to the right-hand rule. Such computation is non elementary for concave polygons, it would be easier to only compute the direction of the normal vector.

Using again $n$ to be the number of vertices of a given polygon, let now $(x_i,y_i,z_i)$ be the coordinates of vertex $V_i$ for $i=1,\ldots,n$. 
Using Newell algorithm (see \cite{newell}), in the case of an embedded polygon in a $3$D space, the oriented normal vector of coordinates $(n_x,n_y,n_z)$ is obtained as follows:
$$
\begin{cases}
n_x=\sum_{i=0}^{n} (y_i-y_{i+1})(z_i+z_{i+1}) \\
n_y=\sum_{i=0}^{n} (z_i-z_{i+1})(x_i+x_{i+1}) \\
n_z=\sum_{i=0}^{n} (x_i-x_{i+1})(y_i+y_{i+1}). 
\end{cases}
$$
In the $2$D case, the algorithm is similar except that we do not need to compute $n_x$ and $n_y$ since they will be set to $0$. 

\subsubsection{Barycentre}
Given a polygon with $n$ vertices, embedded in a $2$D space, let $C$ be its barycenter. With the same notations as in paragraph \ref{onv}, we compute the coordinates $(c_x,c_y)$ of the barycenter as follows \cite{bourke1988calculating}:
$$
\begin{cases}
c_x=\frac{1}{6A}\sum_{i=0}^{n} (x_i+x_{i+1})(x_iy_{i+1}-x_{i+1}y_i) \\
c_y=\frac{1}{6A}\sum_{i=0}^{n} (y_i+y_{i+1})(x_iy_{i+1}-x_{i+1}y_i), \\
\end{cases}
$$
where $A$ is the signed area of the polygon, given by
$$A=\frac{1}{2} \sum_{i=1}^n (x_iy_{i+1}-x_{i+1}y_i). $$

\noindent The case of a polygon embedded in a $3$D space is similar:
\begin{enumerate}
\item If the polygon is embedded in one of the planes parallel to $(O;\mathbf{e_x},\mathbf{e_y})$, $(O;\mathbf{e_x},\mathbf{e_z})$ or $(O;\mathbf{e_y},\mathbf{e_z})$, then the computation is similar to the $2$D case. 
\item Otherwise, we project the polygon on one of the $3$ planes $(O;\mathbf{e_x},\mathbf{e_y})$, $(O;\mathbf{e_x},\mathbf{e_z})$ or $(O;\mathbf{e_y},\mathbf{e_z})$ as in step \ref{pt1area3d} of the algorithm described in \ref{area3d}, and then we proceed as in the $2$D case to determine $2$ out of the $3$ coordinates. To determine the last coordinate, it is projected on one of the two other planes and the procedure is the same. 
\end{enumerate}

\subsection{Polyhedron} \label{polyhedron}
In our implementation, a polyhedron is defined as a \textit{non-ordered} set of polygons. Unlike the implementation of polygons, no algorithm to compute the exact barycenter of polyhedrons has been implemented. Indeed, such value is not necessary for this project, and its implementation would just lose in a useless manner some computational time. It is instead necessary to determine the following characteristics of any polyhedron:
\begin{itemize}
\item its volume, 
\item if the normal vectors of the faces are all coherently oriented, 
\item if the coherently oriented normal vectors are all external or internal to the polyhedron. 
\end{itemize}

\subsubsection{Volume}
To compute the volume of a polyhedron $\Omega$ with exterior normal vector $\mathbf{n}=(n_x,n_y,n_z)$, we use the divergence theorem
$$ \int_\Omega \nabla \cdot F \, \mathrm{d}\Omega=\int_{\partial\Omega}F \cdot\mathbf{n}
\, \mathrm{d}s, $$
with $F(x,y,z) = x$ for all $(x,y,z)\in \Omega$. So that:
$$
\int_\Omega \, \mathrm{d}\Omega=\int_{\partial\Omega} \mathbf{e_x} \cdot \mathbf{n} \, \mathrm{d}s =\int_{\partial\Omega} x  n_x \, \mathrm{d}s.
$$
If $k$ is the number of faces of the polyhedron, and if $\{A_i : 1\leq i\leq k\}$ is the set of its faces, then the second integral can be decomposed into integrals over each $A_i$:
$$ \int_{\partial\Omega} x n_x \, \mathrm{d}s=\sum_{i=1}^k \int_{A_i} x n_x \, \mathrm{d}s.$$
Now, for each $1\leq i\leq k$, let $|A_i|$ be the area of the polygon $A_i$, and $x_i$ be the $x$-coordinate of the barycenter of $A_i$. Then, the volume $V$ of the given polyhedron is:
$$
V = \int_\Omega \, \mathrm{d}\Omega = \sum_{i=1}^k \int_{A_i} x n_x \, \mathrm{d}s = \sum_{i=1}^k |A_i|x_in_x.
$$
Since the area, the barycenter and the normal vector to every face (polygon) of the polyhedron are known (see section \ref{polygons}), then this formula is fast to implement.

\subsubsection{Coherent orientation of the normal vectors of the faces}
Given a polyhedron, we have seen in paragraph \ref{onv} that it is possible to compute oriented normal vectors of all of its faces. However, we now want the normal vectors of all of its faces to be coherently oriented, that is we want all of them to be either internal or external to the polyhedron, but not part of them internal and the rest external. To do so, the algorithm used is the following:
\begin{enumerate}
\item Randomly order the faces of the polyhedron. 
\item For every face, check if it has a common edge with any other face coming after it in the ordering.
\begin{itemize}
	\item If two faces do have a common edge, and if the points defining the common edge appear in an opposite order for both representations of this common edge, then the orientation of the normal vectors is coherent for these two faces (both normal vectors are either external or internal to the polyhedron). 
	\item Otherwise, if two faces do have a common edge but the points appear in the same order for both representations of this common edge, then the order of the vertices in the second polyhedron face's edge has to be inverted to get a coherent orientation. Then, modify the order of the faces in order to have the second face right after the original one in the ordering. 
	\item If a face does not have any common edge with any face coming after it in the ordering, continue without any further operation.
\end{itemize}
\end{enumerate}

\subsubsection{Exterior normal vectors to the polyhedron}
Once we have obtained a coherent orientation of the normal vectors of the faces of the polyhedron, it is necessary to make sure they are all external to the polyhedron. And if it is not the case, we have to inverse all the orientations. This step is necessary to correctly compute and take into account the boundary terms in numerical methods. 

The algorithm that has been implemented is a $3$D version of the \textit{Ray casting algorithm} (see \cite{rayCasting}). In $2$D, this algorithm determines if a point is external or internal to a polygon. To obtain this information, we count the number of intersections between the boundary of the polygon and the segment from an external point to the point we are interested in. If this number is even, the point is external to the polygon. If it is odd, then the point is internal to the polygon. This method can be extrapolated to the $3$D case, but it requires high computational costs. 

There is a problem both in the $2$D and in the $3$D cases. In $2$D, this is when an intersection happens at a vertex instead of at an edge (here, we intend it as without its extremities that are the vertices of the polygon/polyhedron); in $3$D, this is when the intersection happens at a vertex or at an edge, instead of at a face (intended as an open set). In this case, to obtain the desired result, it is necessary to modify the external point that we consider (the $3$D generalization of this affirmation is detailed in the algorithm behind). 

Since all the faces are coherently oriented, it is sufficient to correctly orientate only one of the faces' normal vector. If we had to modify the chosen face's normal vector, then all the other ones, corresponding to the other faces, need to be inverted too. Here is the algorithm used:
\begin{enumerate}
\item Consider one face of the given polyhedron, and a point randomly chosen on it. 
\item Consider the normal vector of this face starting from this point, and extend it to a half-line. 
\item If such extension intersects a vertex or an edge of the polyhedron, then randomly select an other point on the face and repeat the first steps.
\item Count the number of intersections of this half-line with the faces of the polyhedron (without counting the initial chosen point). This is the most expensive step, from a computational point of view. 
\item If the number is even, then the normal vector is external to the polyhedron; if it is odd, then the normal vector is internal to the polyhedron, and the orientation of all faces of the polyhedron has to be inverted.
\end{enumerate}


